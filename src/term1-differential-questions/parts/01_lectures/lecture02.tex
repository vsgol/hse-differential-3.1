\Subsection{Лекция 2}

\begin{theorem}[Арцеоа-Асколи]\thmslashn 
	
	$\{f_n\}; \,\,f_n \in C[a, b]$
	
	$f_n$ равномерно ограниченны и равност. непрерывны
	
	Тогда $\exists f_{n_k} \to \varphi \in C[a, b]$ 
	
\end{theorem}

\begin{lemma}[2]\thmslashn
	
	Если $M \in C[a, b]$
	
	$\exists K: |f(x)| < K\quad \forall x \in [a, b], \forall f \in M$
	
	$\forall \varepsilon > 0\, \exists \delta > 0:$ если $|x_ 1 - x_2| < \delta$, то $|f(x_1) - f(x_2)| < \varepsilon\,\forall f \in M$
	
	То $\exists$ конечная $\varepsilon$-сеть $\forall \varepsilon > 0$
	
\end{lemma}

\begin{remark}\thmslashn

	Если мы докажем лемму 2, то с помощью леммы 1 можно доказать теорему

\end{remark}

\begin{proof}\thmslashn
	
	Фиксируем $\varepsilon > 0, \, \varepsilon = \frac{k}{N}$
	
	Строим решетку с шагом $\delta$ по горизонтали и с шагом $\varepsilon$ по вертикали
	
	Берем все ломанные по сетке и заметим, что их конечное кол-во. 
	
	Докажем, что ломанные являются $\varepsilon$-сетью для функций из $M$ 
	
	$f \in M$. Пусть $\psi_k$ -- ближайший узел для $f(x_k)$.
	
	Рассмотрим $\psi(x)$ -- ломанная: $\psi(x_k) = \psi_k$
	
	Докажем, что $\rho(f, \psi) < 5\varepsilon$
	
	$(?)\, \max_{x\in[a, b]}|f(x) - \psi(x)| < 5\varepsilon$
	
	$\forall m \quad |f(x_m) - \psi(x_m)| < \varepsilon$ по определению $\psi$
	
	$\forall x \in [x_m, x_{m+1}]:\,\, |f(x) - \psi(x)| \leqslant |f(x) - f(x_m)| + |f(x_m) - \psi(x)| \leqslant \varepsilon + |f(x_m) - \psi(x_m)| + |\psi(x_m) - \psi(x)| \leqslant 2\varepsilon + |\psi(x_m) - \psi(x_{m+1})| \leqslant 2\varepsilon + |\psi(x_m) - f(x_m)| + |f(x_m) - f(x_{m+1})| + |f(x_{m+1}) - \psi(x_{m+1})| \leqslant 5\varepsilon$
	
	Успех

\end{proof}

\begin{proof}[теоремы А-А]\thmslashn
	
	По лемме 1 если существует конечная $\varepsilon$-сеть $\forall \varepsilon$, то $\exists \varphi$
	
\end{proof}

\begin{exerc}\thmslashn
	\begin{enumerate}
		\item 
		К лемме 2: доказать, что обратное тоже верно

		\item
		К Т. А-А: доказать, что обратное тоже верно

	\end{enumerate}
\end{exerc}

\begin{remark_author}\thmslashn
	\[
	\begin{cases}
		y'(x) &= F(x, y(x))\\
		y(x_0)&= y_0\\
	\end{cases}
	\]
	$F\in C(D), y_0\in D$

	Идея решения
	
	\TODO Блин я не записал(((

\end{remark_author}

\begin{properties}\thmslashn

	$Q = \{(x, y) | |x - x_0| \leqslant A, |y - y_0| \leqslant B\} \subset D$

	$Q$ -- компакт $\Rightarrow \max_{Q} F(x, y) = M$ 

	$n = \min\left(A, \frac{B}{M}\right)$
	\begin{enumerate}
		\item
		Тогда на $[x_0, x_0 + h]$ ломаные Эйлера для некоторого $\delta = \frac{n}{N}$ остаются внутри $Q$ за $N$ шагов
		
		\begin{proof}\thmslashn
			
			Все $x_k: |x_k - x_0| \leqslant A$
			
			$|y_k - y_0| \leqslant |y_k - y_{k-1}| + \ldots + |y_1 - y_0| = |F(x_{k-1} - y_{k-1})| \cdot |x_k - x_{k-1}| + \ldots + |F(x_0, y_0)|\cdot|x_{1} - x_0| \leqslant M(|x_{k} - x_{k-1}| + \ldots + |x_1 - x_0|) = M\cdot \delta \cdot k \leqslant M \cdot \delta \cdot N \leqslant M\cdot n \leqslant B$
			
		\end{proof}
		
		\item
		$\forall \varepsilon > 0$ для достаточно большого $N$ ломанная Эйлера для $\delta = \frac{n}{N}$
		
		\begin{enumerate}
			\item
			$(x, \psi(x)) \in D\,\,\forall x \in [x_0, x_0 + h]$
			
			\item
			$|\psi'(x) - F(x, \psi(x))| < \varepsilon \,\,\forall x \in [x_0, x_0 + h]$ кроме точек дробления

		\end{enumerate}

		\begin{proof}\thmslashn

			Пункт $1)$ есть по лемме 1 (даже $\in Q$)

			Пункт $2)$ $F \in C(Q), \,\, Q$ -- компакт $\Rightarrow F$ равномерно непрерывна

			$\exists \delta_1:$ если $|x_1 - x_2| < \delta_1$ и $|y_1 - y_2| < \delta_1$, то $|F(x_1, y_1) - F(x_2, y_2)| < \varepsilon$

			Выберем $N, \,\,\delta = \frac{n}{N}$, что $\delta < \frac{\delta_1}{M}$ и $\delta < \delta_1$

			Пусть $x_0, x_1, \ldots, x_k, \ldots, x_n$ -- точки дробления

			$\psi'(x) = F(x_k, y_k)$ для $x \in [x_k, x_{k+1}]$

			$\psi'(x) = |F(x_k, \psi(x_k)) - F(x, \psi(x))| < \varepsilon$ 

			Надо доказать, что разность аргументов $F(x_k, \psi(x_k))$ и $F(x, \psi(x)$ меньше, чем $\delta_1$

			$|x_k - x| < \delta \leqslant \delta_1;\,\,|\psi(x_k) - \psi(x)| < |\psi(x_k) - \psi(x_{k+1})| < M \cdot \delta \leqslant \delta_1$

		\end{proof}

		\item
		$f_n \to \varphi$ равномерно

		Тогда $F(x, f_n(x)) \to F(x, \phi(x))$ равномерно на $[a, b]$

		\begin{proof}\thmslashn

			$|F(x, y_1) - F(x, y_2)| < \varepsilon$

			при $|y_1 - y_2| < \delta$ (т.к. $F$ равномерно непрерывна на $Q$)

			$f_n \to \phi$ равномерно $\Leftrightarrow \max_{x\in[x_0, x_0 + h]} |f_n(x) - \phi(x)| \to 0$ при $n \to \infty$

			$\forall \delta > 0\,\, \exists N: \,\, \forall n > N\quad |f_n(x) - \phi(x)| < \delta\, \forall x \in [x_0, x_0 + h]$

			Итого: $\forall \varepsilon > 0\, \exists N: \, \forall n > N\quad |F(x, f_n(x)) - F(x, \phi(x))| < \varepsilon \, \forall x \in [x_0, x_0 + h]$

		\end{proof}

		\begin{remark}\thmslashn

			Все рассуждения повторяются на $[x_0 - h, x_0]$

		\end{remark}
	\end{enumerate}
\end{properties}

\begin{theorem}[Пеано]\thmslashn 

	Пусть $A, B, Q, M$ и $h$ как ранее
	
	Тогда на $[x_0 - h, x_0 + h]$ существует решение задачи Коши

\end{theorem}

\begin{proof}\thmslashn

	Возьмем убывающую последовательность $\varepsilon_k \to $. Строим $\psi_k$ по лемме 2 для $\varepsilon = \varepsilon_k$ 
	
	\begin{enumerate}
		\item
		$\psi_k$ -- равномерно ограничена:

		$|\psi_k(x) - y_0| \leqslant B \Rightarrow |\psi_k(x)| \leqslant B + |y_0|$

		\item
		$\psi_k$ -- равностепенно непрерывны:
		
		$|\psi_k(x) - \psi_k(\~{x})| \leqslant M|x - \~{x}|$ если $x, \~{x} \in [x_m, x_{m+1}] \Rightarrow \forall x, \~{x} \in [x_0 - h, x_0 + h]$
		
		$\delta$ по $\epsilon$ строится как $\delta = \frac{\varepsilon}{M}$
		
		Тогда по теореме А-А $\exists \psi: \psi_k \to \psi$ раномерно на $[x_0 - h, x_0 + h]$
		
		Докажем, что $ \psi(x) = y_0 + \int\limits_{x_0}^x F(x, \psi(x))\,dx$
		
		$\psi_k(x) = y_0 + \int\limits_{x_0}^x F(x, \psi_k(x))\,dx + \int\limits_{x_0}^x \omega_k(x)\,dx$, т.к. $\phi_k(x) = \psi_k(x_0) + \int\limits_{x_0}^x \psi_k'(x)\,dx$
		
		где $\omega_k(x) = \psi_k'(x) - F(x, \psi(x))$  кроме точке дробления
		
		$\int\limits_{x_0}^x \omega_k(x)\,dx \to 0$
		
		$\abs{\int\limits_{x_0}^x \omega_k(x)\,dx} \leqslant \int\limits_{x_0}^x \abs{\omega_k(x)}\,dx \leqslant \varepsilon_k \cdot |x - x_0| \leqslant \varepsilon_k \cdot n \to 0$ по лемме 2
		
		по лемме 3 $\abs{ F(x, \psi_k(x)) - F(x, \psi(x))} < \varepsilon$ при достаточно больших $k$
		
		Тогда $\abs{\int_{x_0}^x F(x, \psi(x))\,dx - \int_{x_0}^x F(x, \psi_k(x))\,dx } \leqslant \int_{x_0}^x \abs{F(x, \psi(x)) - F(x, \psi_k(x))}\,dx < \varepsilon \cdot |x - x_0| < \varepsilon \cdot h \to 0 $
		
		Получаем, что $\psi$ -- решение задачи Коши

	\end{enumerate}

\end{proof}
